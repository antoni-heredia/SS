\documentclass[12pt,a4paper]{article}
\usepackage[spanish,es-tabla]{babel}
\usepackage{float}							% Insertar figuras
\usepackage{subcaption}

\usepackage[utf8]{inputenc} % Escribir con acentos, ~n...
\usepackage{eurosym} % s´ımbolo del euro
\newcommand{\horrule}[1]{\rule{\linewidth}{#1}} % Create horizontal rule command with 1 argument of height
\usepackage{listings}             % Incluye el paquete listing
\usepackage[cache=false]{minted}
\usepackage{graphics,graphicx, float} %para incluir imágenes y colocarlas
\usepackage{hyperref}
\hypersetup{
	colorlinks,
	citecolor=black,
	filecolor=black,
	linkcolor=black,
	urlcolor=black
}
\usepackage{multirow}
\usepackage{array}
\usepackage{diagbox}
\usepackage{listings}

\lstset{language=Java,
	keywordstyle=\color{RoyalBlue},
	basicstyle=\scriptsize\ttfamily,
	commentstyle=\ttfamily\itshape\color{gray},
	stringstyle=\ttfamily,
	showstringspaces=false,
	breaklines=true,
	frameround=ffff,
	frame=single,
	rulecolor=\color{black}}
\title{
\normalfont \normalsize 
\textsc{{\bf Simulación de Sistemas (2019-2020)} \\ Grado en Ingeniería Informática \\ Universidad de Granada} \\ [25pt] % Your university, school and/or department name(s)
\horrule{0.5pt} \\[0.4cm] % Thin top horizontal rule
\huge Práctica 1 \\ % The assignment title
\horrule{2pt} \\[0.5cm] % Thick bottom horizontal rule
\includegraphics{images/logo.png}	
}

\author{Antonio Jesús Heredia Castillo} % Nombre y apellidos

\date{\normalsize\today} % Incluye la fecha actual

%----------------------------------------------------------------------------------------
% DOCUMENTO
%----------------------------------------------------------------------------------------

\begin{document}

\maketitle % Muestra el Título
\newpage %inserta un salto de página
\tableofcontents % para generar el índice de contenidos
\newpage
\section{Mi Primer Modelo de Simulación de MonteCarlo}
En este simulador tenemos como variables de entrada las siguientes variables:
\begin{itemize}
	\item \textbf{Numero de iteraciones}: la cantidad de repeticiones que se va ejecutar el simulador
	\item \textbf{Posición de destino:} La posición donde quiere aparcar el usuario.
	\item \textbf{Numero de posiciones a la vista:} La cantidad de posiciones que puede ver hacia delante el conductor desde el coche.
	\item \textbf{Probabilidad de plaza ocupada: } Con que probabilidad se va a encontrar la plaza ocupada.
\end{itemize}

Dependiendo de las variables de entrada que especifiquemos tendremos unos datos de salida diferentes. \\
La primera prueba que realizaremos  es ejecutar la misma simulación varias veces para ver como afecta la la generación de numeros aleatorios.

\begin{table}[H]
	\centering
	\begin{tabular}{|l|l|l|l|}
		\hline
		\textbf{Variable entrada} & \textbf{Valor}\\ \hline
		Numero de iteraciones&100000\\ \hline
		Posición de destino&100\\ \hline
		Numero de posiciones a la vista&2\\ \hline
		Probabilidad de plaza ocupada&0.9\\ \hline
	\end{tabular}
\end{table}
Ejecutaremos varias veces la simulación con estos datos y veremos en una tabla comparativa como afecta. 
\begin{table}[H]
	\centering
	\begin{tabular}{|l|l|l|l|}
		\hline
		\textbf{Nº simulación} & \textbf{Mejor posición inicial}&\textbf{Mejor distancia}\\ \hline
		1&94&6.47\\\hline
		2&94&6.55\\\hline
		3&95&6.48\\\hline
		4&94&6.48\\\hline
		5&94&6.54\\\hline
		6&95&6.50\\\hline
	\end{tabular}

\end{table}
Como podemos ver la simulación varia muy poco en las diferentes. Esto se puede ver mejor en la Figura \ref{fig:mediadistancia}. En la que todos los valores están entre $6.4$ y $6.6$.
\begin{figure}[H]
	\centering
	\includegraphics{images/mediaDistancia}
	\caption{Distancia media en las diferentes simulaciones.}
	\label{fig:mediadistancia}
\end{figure}
Aunque de forma intuitiva podemos ver como a pesar de tener una probabilidad muy alta de que la plaza este ocupada siempre encuentra aparcamiento a una distancia que podemos considerar cercana. Veremos si esto se cumple cambiando los parámetros de entrada y haciendo diferentes simulaciones.

\subsection{Variando la posición de destino}
Para ver si afecta la a la distancia la posición a la que se quiere ir ejecutaremos varias veces la simulación con distintas posiciones final. Aunque intuitivamente podemos pensar que no, lo comprobaremos. \\ Ejecutaremos la simulación con las siguientes variables:
\begin{table}[H]
	\centering
	\begin{tabular}{|l|l|l|l|}
		\hline
		\textbf{Variable entrada} & \textbf{Valor}\\ \hline
		Numero de iteraciones&100000\\ \hline
		Posición de destino&97:103\\ \hline
		Numero de posiciones a la vista&2\\ \hline
		Probabilidad de plaza ocupada&0.9\\ \hline
	\end{tabular}
\end{table}
En la Figura \ref{fig:distanciaPosFinal} podemos observar que lo que intuíamos se cumple. Podemos ver como la curva de disminución de la distancia media en función de la posición inicial donde comienza a buscar aparcamiento es igual pero desplazada una posición. Por lo tanto  en vista de los resultados podemos confirmar que la posición final no afecta a la mejor distancia para encontrar aparcamiento.
\begin{figure}[H]
	\centering
	\includegraphics{images/distanciaPosFinal.png}
	\caption{Distancia media en relación a la posición final.}
	\label{fig:distanciaPosFinal}
\end{figure}
\subsection{Variando el numero de posiciones a la vista}
En este caso podemos pensar que cuantas mas posiciones ves hacia delante mejor podrás  predecir cuando escoger un aparcamiento y cuando no y asi mejorar la distancia a la que aparcar. Veremos si se cumple o en clase de que se cumpla si merece la pena anticiparse.\\
En este caso ejecutaremos la simulación con las siguientes variables:
\begin{table}[H]
	\centering
	\begin{tabular}{|l|l|l|l|}
		\hline
		\textbf{Variable entrada} & \textbf{Valor}\\ \hline
		Numero de iteraciones&100000\\ \hline
		Posición de destino&100\\ \hline
		Numero de posiciones a la vista&2,12,22,32\\ \hline
		Probabilidad de plaza ocupada&0.9\\ \hline
	\end{tabular}
\end{table}
Para analizar los datos usaremos la Figura \ref{fig:distanciaVisibilidad} que la visualizaremos en tres dimensiones, aunque no nos importa mucho en que posición inicial empezamos a buscar aparcamiento. Como podemos ver con visibilidad $0$  la distancia de aparcado esta entorno a $6.5$ y conforme aumentamos la visibilidad disminuye fuertemente. hasta que llega entorno al $4.75$ y ya apenas varia. Con estos datos podemos deducir que cuanto mas coches hacia delante intente ver el conductor mas cerca podra aparcar de su destino. 
\begin{figure}[H]
	\centering
	\includegraphics{images/distanciaVisivilidad.png}
	\caption{Distancia media en relación a la visibilidad final.}
	\label{fig:distanciaVisibilidad}
\end{figure}
\subsection{Variando la probabilidad de plaza ocupada}
En este caso cambiaremos la probabilidad en que una plaza este libre o no. Como es de esperar cuanto mas probable sea que la plaza este ocupada mas difícil sera encontrar aparcamiento y por ende mas lejos se tendrá que aparcar.\\La simulación la he realizado con los siguientes datos.
\begin{table}[H]
	\centering
	\begin{tabular}{|l|l|l|l|}
		\hline
		\textbf{Variable entrada} & \textbf{Valor}\\ \hline
		Numero de iteraciones&100000\\ \hline
		Posición de destino&100\\ \hline
		Numero de posiciones a la vista&2\\ \hline
		Probabilidad de plaza ocupada&0.5,0.6,...,0.9\\ \hline
	\end{tabular}
\end{table}
Según la Figura \ref{fig:distProbabilidad} podemos observar que la distancia en función de la probabilidad de que este ocupada tiene un crecimiento exponencial y afecta mucho a la hora tener una distancia mejor o peor. 
\begin{figure}[H]
	\centering
	\includegraphics{images/probabilidad.png}
	\caption{Distancia media en relación a la visibilidad final.}
	\label{fig:distProbabilidad}
\end{figure}
En cambio la distancia a la que deberíamos empezar a buscar aparcamiento si que varia de forma linea como podemos ver en la Figura \ref{fig:distProbabilidad2}.
\begin{figure}[H]
	\centering
	\includegraphics{images/prob2.png}
	\caption{Distancia media en relación a la visibilidad final.}
	\label{fig:distProbabilidad2}
\end{figure}
Si nos pusieramos en el caso de que fuera imposible encontrar aparcamiento porque todas las plazas van a estar llenas el simulador fallaría como podemos ver la Figura \ref{fig:falloej1}.
\begin{figure}[H]
	\centering
	\includegraphics[width=0.7\linewidth]{images/falloej1}
	\caption{}
	\label{fig:falloej1}
\end{figure}


\end{document}
	